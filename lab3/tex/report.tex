\documentclass[a4paper,12pt]{article}

\usepackage{graphicx}
\usepackage{wrapfig}
\usepackage{amsmath}
\usepackage{amsfonts}
\usepackage[english, russian]{babel}
\usepackage[T1, T2A]{fontenc}
\usepackage[utf8]{inputenc}
\usepackage{geometry}
\usepackage{indentfirst}
\usepackage{listings}
\usepackage[dvipsnames]{xcolor}
\usepackage[colorlinks]{hyperref}
\usepackage{amsmath}
\graphicspath{ {./images/} }
\geometry{left=2cm, right=2cm, bottom=2cm, top=2cm}

\hypersetup{
    linkcolor=blue,
    filecolor=magenta,
    urlcolor=cyan
}

\lstset{
    backgroundcolor=\color{white},
    basicstyle=\footnotesize\ttfamily,
    breaklines=true,
    frame=single,
    numbers=left,
    numberstyle=\tiny\color{gray},
    keywordstyle=\color{blue},
    commentstyle=\color{green!40!black},
    stringstyle=\color{darkblue},
    showstringspaces=false,
    tabsize=4,
    language=Mathematica
}

\definecolor{codegreen}{rgb}{0,0.6,0}
\definecolor{codegray}{rgb}{0.5,0.5,0.5}
\definecolor{codepurple}{rgb}{0.58,0,0.82}
\definecolor{backcolour}{rgb}{0.95,0.95,0.92}


\lstdefinestyle{codestyle}{
    backgroundcolor=\color{backcolour},
    commentstyle=\color{codegreen},
    keywordstyle=\color{blue},
    numberstyle=\tiny\color{codegray},
    stringstyle=\color{magenta},
    basicstyle=\ttfamily\footnotesize,
    breakatwhitespace=false,
    breaklines=true,
    captionpos=b,
    keepspaces=true,
    numbers=left,
    numbersep=5pt,
    showspaces=false,
    showstringspaces=false,
    showtabs=false,
    tabsize=2
}













\lstset{style=codestyle}
\lstset{extendedchars=\true}

\begin{document}
\begin{titlepage}
    \centering
    \vspace*{1cm}

    {\large Министерство науки и высшего образования Российской Федерации}\\
    {\large ФЕДЕРАЛЬНОЕ ГОСУДАРСТВЕННОЕ АВТОНОМНОЕ ОБРАЗОВАТЕЛЬНОЕ УЧРЕЖДЕНИЕ ВЫСШЕГО ОБРАЗОВАНИЯ «НАЦИОНАЛЬНЫЙ ИССЛЕДОВАТЕЛЬСКИЙ УНИВЕРСИТЕТ ИТМО»}\\
    {\large (УНИВЕРСИТЕТ ИТМО)}\\

    \vspace{2cm}

    {\large Факультет «Систем управления и робототехники»}\\

    \vspace{3cm}

    \textbf{{\Huge ОТЧЕТ}\\
    {\Huge О ЛАБОРАТОРНОЙ РАБОТЕ №3}}\\

    \vspace{1cm}

    {\LARGE По дисциплине «Частотные методы»}\\
    {\LARGE на тему: «Жесткая фильтрация»}\\

    \vspace{3cm}

    {\Large Студент:}\\
    Охрименко Ева

    \vspace{2cm}

    {\Large Преподаватели:}\\
    Догадин Егор Витальевич\\
    Пашенко Артем Витальевич\\

    \vspace{3cm}
    {\large г. Санкт-Петербург}\\
    {\large 2025}

\end{titlepage}
\newpage
\tableofcontents
\newpage

\section{Task. Жесткие фильтры}
\subsection{Краткое условие}

Рассмотрите функцию \( g(t) \), заданную как:
\[
g(t) = 
\begin{cases} 
a, & t \in [t_1, t_2], \\
0, & t \notin [t_1, t_2],
\end{cases}
\]
и её зашумлённую версию:
\[
u(t) = g(t) + b\xi(t) + c \sin(dt),
\]
где \(\xi(t) \sim U[-1, 1]\) — белый шум, а \( b, c, d \) — параметры.

\begin{itemize}
\item При \( c = 0 \) найдите Фурье-образ \( u(t) \), обнулите его вне \([- \nu_0, \nu_0]\) и выполните обратное преобразование. Исследуйте влияние \( \nu_0 \) и \( b \).

\item При ненулевых \( b, c, d \) обнулите Фурье-образ на выбранных частотах, подавляя шум и гармонику. Исследуйте влияние параметров.

\item бнулите Фурье-образ в окрестности \( \nu = 0 \), пропустите сигнал через фильтр и оцените результат.
\end{itemize}
\textbf{Ожидаемые результаты:} \\
Графики исходного, зашумлённого и фильтрованного сигналов, а также их Фурье-образов. Выводы по каждому пункту.

\subsection{Убираем высокие частоты}
\subsubsection{Предподготовка}
Для начала выберу все нужные параметры для этого задания.






\subsection{Убираем специфические частоты}
\subsection{Убираем низкие частоты?}




\section{Task. Фильтрация звука}
\subsection{Краткое условие}

\end{document}
